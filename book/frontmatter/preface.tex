\chapter*{Preface}
\addcontentsline{toc}{chapter}{Preface}
\vspace{5mm}

\section{How to Use This Book}

\justify{}
The ideas captured here are not a means to any particular end. Rather, these are meant to be starting points, giving
you a frame of reference with novel technologies and techniques to streamline your workflows and give your projects
a productivity boost. The hope is the reader will gain momentum to pursue these new ideas by following along with the
examples outlined in this book.

\justify{}
You should work to build up your own solid base of code examples and problem solving techniques that will greatly increase
your efficacy. Over time, new tools and processes will rotate in and out of your toolbox as technology progresses. Keep
in mind that your job is to maintain that momentum, to keep experimenting and to see what is useful enough to stick with
you and make a permanent part of your technical repertoire.

\justify{}
This book is meant to be a workbook as much as it is meant to be read. You are encouraged to jump ahead, go back and re-read,
do the exercises you think you can apply the learning objective from right away, and skip the parts you don't think you will
ever use. Learning can be a non-linear experience and you are encouraged to ``color outside the lines'' to the extent you feel
comfortable doing so. That said, I've attempted to give this book a feel of moving the reader
along towards a final lab project. This project is meant to guide the reader through application of the topics covered before 
the final chapter.

\justify{}
Companies often make their services free in the hopes that you will see the value and usefulness of their products. Their
thinking goes that hopefully you will see enough utility that you will recommend them to your enterprise clients and
integrate their products into your workflows. Not a bad trade-off! It only makes sense to avail yourself of free-tier
cloud services, build and test platforms, and low or no-cost hosting environments. There are plenty of these out there and
we will explore some of these as we dive further into the topics.

\justify{}
When we choose to use a tool, say Ansible\index{Ansible} for example, we must adopt the
most up-to-date and best practices for using that tool. File system layout, naming conventions, script syntax and organization,
and so on. We get to enjoy the clear and safe path forged by the folks that came before us, and with whom we share many goals.
As your skills mature over time, you may want to consider donating time back to the
open source and other communities upon which your knowledge has been built.

\justify{}
Finally, I find it very helpful for my own personal peace of mind to leave projects ``clean and green'', to the extent
possible. In other words, there is mental benefit to tidying up your physical and virtual workspace
before walking away from the keyboard for the day. Perhaps you would find similar benefit should you choose to adopt
this practice.

\section{Acknowledgments}

\justify{}
Creation is a long and twisty path, fraught with the distractions of a life well-lived and the frenetic pace of a day and
age that clamors for a million tiny bits of our attention. A supportive and loving family is the touchstone that grounds
us through it all. The author would like to thank his family for making it possible to maintain focus in a focus-stealing world.

\justify{}
Feedback has been a key component in getting these words organized into the order they appear before you today. Thank you to
those folks who contributed their time to review this book including Aaron Didier (@phreakinggeek).

\justify{}
A special thank you to {\href{https://www.linkedin.com/in/eddiemize/}{Eddie Mize} for providing the cover art for this book
	and being a good friend.
