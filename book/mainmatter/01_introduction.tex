\chapter{Introduction}

\includegraphics[scale=0.85]{../images/sky-690293_1920.jpg}

\justify
The term DevSecOps\index{DevSecOps} is an amalgam of the words development, security and
operations. Simply put, DevSecOps lives at the intersection of
application development, information security, and network operations.
As we will soon discover, Infrastructure as Code (IaC)\index{Infrastructure as Code (IaC)} is the platform
upon which software flows moves through a cycle of Continuous Deployment or Continuous Delivery
(CD)\index{Continuous Delivery (CD)}. To the extent possible, we attempt to realize gains in performance
of our people and projects by leveraging automation and Agile
development practices throughout this cycle.

\justify
The world is changing with respect to how software is created and maintained. Folks at the leading edge in today's computing industry are not just building software, but are curating it through a cyclical
process of continuous development, testing, use, and improvement. With increasing frequency, applications and workloads are moving to computing
environments that are abstracted away, managed by invisible armies of engineers at companies other than their own. Of course we are referring
to those multitenant cloud type computing landscapes. Passing one or more fully encapsulated applications to a cloud provider for the purposes of having them host it as a production environment has become
commonplace. Further, cloud service providers are adding new features and capabilities at breakneck speed.

\justify
At the time of this writing in 2020, about 40\% of production workloads are running on containers and serverless deployments. Bare metal and virtual machines currently host a bit over 60\% of production workloads.
Containerized workload use is expected to increase even more in the coming years. Conversely, bare metal and VM usage is expected to decrease\footnote{\url{https://start.paloaltonetworks.com/esg-research-cloud-native-devsecops-report.html}}. It's not a question of if, but how quickly commoditization of compute resources takes place, perhaps leaving only a few main providers of these cloud resources. This is not unlike how power generation and
distribution became centralized in the previous century, now the domain of a few large utility companies. Nothing beyond considerations of time, money, and practicality stop you from making your own electricity, but
most folks are keen to invest their efforts in other pursuits.

\justify
In this book, we will explore a combination of techniques that can refresh your skills and align your projects with the technological leading edge. We will introduce various popular technologies, then use
common bits and pieces of these to create a secure build pipeline for our lab and development work, test, and even production environments. The techniques here are meant to help the security-minded developer
sharpen her or his skills, and introduce tips and tactics that benefit the teams they are a part of. There are many, many ways to reach similar goals these days with the preponderance of Open Source and commercial tools that are available. By focusing on a few we can blaze a trail to success in our projects.

\justify
We have a goal in mind of selecting complementary tools and process to construct and streamline our ways of working. We will attempt to leverage these ways to get us quickly and securely to a working lab
environment. At the same time we should strive for simplicity and reduction of complexity whenever possible. Experience tells us that tools and process that are too cumbersome or burdensome are typically
circumvented, or even abandoned. Complexity in our processes become the snags and side projects that are the enemy of productivity. Refuse to shave more yaks than absolutely necessary!
