\chapter{Continuous Integration \& Deployment}

\includegraphics[scale=0.85]{../images/finland-905712_1920.jpg}

\justify
Accommodations for Continuous Integration (CI) and Continuous Deployment (CD) in our projects directly corresponds to our chances of success.

\section{Linters}

\justify
There are small programs for most (every?) language that you can run before pushing your changes to GitHub that will catch syntactical and sometimes even programmatic issues. Consider Python, which is very sensitive with regard to indentation. You can programatically detect and even correct issues before your work gets too far down the pipe. This is also a good way to make sure folks are not committing dirty code to your repositories.

\justify
Here are some of the linters I have found useful for languages I encounter frequently.

\begin{table}[h]
\begin{center}
\begin{tabular}{| p{2.5cm}| p{2.5cm} | p{2.5cm} |}
	\hline
	\textbf{Language} & \textbf{Name} & \textbf{Source}\\
	Ansible & ansible-lint & python (pip install ansible-lint)\\
	Markdown & mdl & ruby (gem install mdl)\\
	Puppet & puppet-lint & ruby (gem install puppet-lint)\footnote{\url{http://puppet-lint.com/}}\\
	Python & pylint/flake8 & python (pip install pylint/flake8)\\
	Terraform & tflint & \url{https://github.com/terraform-linters/tflint}
\end{tabular}
\end{center}
\end{table}

\subsection{Linting with Tox}

Recall that we are using Tox as our main test framework. To set up Tox
to do our linting work for us, we can add an environment to our envlist
called "pylint" and then declare it in a new stanza in tox.ini. Notice
how we let "deps" do the work of installing the "pylint" dependency for
us.

\begin{mybox}{\thetcbcounter: A Makefile Target}
	\lstinputlisting{code/toxfile-pylint}
\end{mybox}

\section{GitHub Actions}

\justify
GitHub Actions is a recent introduction to the github.com website that
lets you perform Continuous Integration on your repository, and Continuous Deployment as desired.

\subsection{Docker}

Let's see how we can leverage Actions to build the docker target in our
project. Save this YAML file under
codelab/.github/workflows/docker\_compose.yml to have GitHub Actions
execute our make docker target from our custom Makefile.

\begin{mybox}{\thetcbcounter: A Makefile Target}
	\lstinputlisting{code/makefile-target}
\end{mybox}


\section{Python}

Save this YAML file under codelab/.github/workflows/python.yml to have
GitHub Actions execute our make python target from our custom Makefile.

\begin{mybox}{\thetcbcounter: A Makefile Target}
	\lstinputlisting{code/makefile-target}
\end{mybox}

\subsection{Packer}

Save this YAML file under devsecops/.github/workflows/packer.yml to have
GitHub Actions validate and build our AMI image with Packer.


\begin{mybox}{\thetcbcounter: A Makefile Target}
	\lstinputlisting{code/makefile-target}
\end{mybox}

\subsection{Markdown}

The following example YAML file illustrates how to validate GitHub
flavored Markdown text files using a GitHub Action.

\begin{mybox}{\thetcbcounter: A Makefile Target}
	\lstinputlisting{code/makefile-target}
\end{mybox}

\justify
Note the designation of a configuration file named .markdownlint.json at
the top level of our repository. This JSON file is used to skip certain
checks by the markdownlint tool.

\justify
\begin{mybox}{\thetcbcounter: A Makefile Target}
	\lstinputlisting{code/makefile-target}
\end{mybox}

\section{Circle CI}

Circle CI is a Continuous Integation service free for non-commercial
projects.

single: Circle CI

\justify
\begin{mybox}{\thetcbcounter: A Makefile Target}
	\lstinputlisting{code/makefile-target}
\end{mybox}

\section{TravisCI}

\justify
Travis CI is a hosted continuous integration service used to build and
test software projects hosted at GitHub and Bitbucket. They have a great
tutorial available\footnote{\url{https://docs.travis-ci.com/user/tutorial/}}
if you care to dig a bit deeper.

\justify
By enabling Travis CI integration through the GitHub
Marketplace\footnote{\url{https://github.com/marketplace/travis-ci}} you
can integrate their scanners with your repository.

\subsubsection{Docker}

\justify
You can test Docker containers in your CI/CD pipeline. As seen in the
following example I created a YAML file named .travis.yml to enable
automatic molecule testing of ansible roles in Travis CI. I also set a
flag in the repo settings that prevent the PR from being merged until
Travis CI flags the build as passing.

\justify
\begin{mybox}{\thetcbcounter: A Makefile Target}
	\lstinputlisting{code/makefile-target}
\end{mybox}

The contents of the requirements files and the example Ansible code is
available in the companion repo.


\subsection{Markdown}

Save these lines to a file named .travis.yml to scan all the markdown files in your repository.

\justify
\begin{mybox}{\thetcbcounter: A Makefile Target}
	\lstinputlisting{code/makefile-target}
\end{mybox}

\justify
You can also create an .mdlrc file to give mdl direction on what to scan for.

\justify
\begin{mybox}{\thetcbcounter: A Makefile Target}
	\lstinputlisting{code/makefile-target}
\end{mybox}

\clearpage

\section{Directory Structure}

\justify
Relevant folders and files related to our build pipeline are shown below. The users home directory and workspace subdirectory is implied and removed from the diagram for clarity.



%"devsecops" -\textgreater{}
%".github"; "devsecops" -\textgreater{} ".mdlrc"; 
%"devsecops"-\textgreater{} ".travis.yml"; 
%"devsecops" -\textgreater{} ".markdownlint.json";
%"devsecops" -\textgreater{} ".circleci";
%".circleci" -\textgreater{} "config.yml";
%".github" -\textgreater{} "workflows";
%"workflows" -\textgreater{} "docker\_compose.yml";
%"workflows" -\textgreater{} "markdown.yml"; "workflows" -\textgreater{}
%"packer.yml"; "workflows" -\textgreater{} "python.yml";


\begin{figure}[!htb]
	\centering
	\digraph{cicd}{
		node [shape=rectangle];
		
		% nodes
		devsecops [label="devsecops"]; 
		dotgh [label=".github"];
		workflows [label="workflows"]; 
		dotcircleci [label=".circleci"];
		cy [label="config.yml"];
		dcy [label="docker\_compose.yml]"; 
		py [label="packer.yml]"; 
		pyyml [label="python.yml"];
		mdy [label="markdown.yml"];
		dottr [label=".travis.yml"];
		mdlrc [label=".mdlrc"];
		mdjson [label=".markdownlint.json"];
    }
	\caption{Project Directory and related files.}
\end{figure}